\chapter*{Samenvatting}
\addcontentsline{toc}{chapter}{Samenvatting}

{\selectlanguage{dutch}

% What is the thesis about?
% What is the purpose of the thesis?
% What were the methods used to research the information?
% What are the results, conclusions, and recommendations that the thesis presents?


\noindent 
\dropcap{S}{oortvorming} is een biologisch proces waarbij
nieuwe soorten ontstaan. Met een fylogenetische boom
kun je de evolutionare historie en verwantschap van soorten weergeven.
Deze bomen kunnen niet direct gemeten worden. Inplaats daarvan
worden ze berekend. Achter deze berekening schuilt een wiskundig model
met veel aannames, waaronder de aanname hoe soortvorming werkt.

In die proefschift kijk ik naar het effect van de aanname van hoe soortvorming
werkt. We denken van de meestgebruikte soortvormingsmodellen dat ze simpel
genoeg zijn, maar niet t\'{e} simpel. Ik meet in hoeverre dat klopt.

In hoofdstuk 2 liet ik \verb;babette; zien: een R package waarmee je BEAST2, 
een Bayesiaans phylogenetisch inferentieprogramma, kunt aanroepen. 
\verb;babette; is een flexibel en robuust programma geworden.

In hoofdstuk 3 liet ik \verb;pirouette; zien: een R package waarmee
je kunt meten hoe groot de invloed is van een juist of onjuist 
soortvormingsmodel. Ik en mijn mede-auteur laten zien dat dit instrument naar behoren werkt, mits
je genoeg herhalingen doet om toevalseffecten een kleine rol te laten spelen.

In hoofdstukken 4 en 5 liet ik zien wat de invloed is van het gebruik van een standaard 
soortvormingsmodel als het echte soortvormingsproces stiekum iets complexer is.
In hoofdstuk 4 zijn de echte fylogenetische bomen gesimuleerd met een 
soortvormingsproces waarbij w\'{e}l twee soorten tegelijkertijd kunnen ontstaan.
Ik en mijn mede-auteur laten zien dat ...

In hoofdstuk 5 zijn de echte fylogenetische bomen gesimuleerd met een 
soortvormingsproces waarbij soortvorming w\'{e}l tijd kost. Ik laat zien
dat ... en dat het het effect van [sampling] ... is.  

Dit proefschrift leert ons dat ...

Het mooie aan mijn onderzoek is dat andere wetenschappers er zelf gemakkelijk 
ook wat mee kunnen:
zowel \verb;babette; als \verb;pirouette; zijn flexibele en professionele 
R packages. \verb;babette; heeft het mogelijk gemaakt om grootschaliger 
onderzoek te doen aan fylogenetische modellen, doordat nu een experiment
vanuit een script gedaan kan worden, inplaats van handmatig elk 
experiment in te stellen. Met \verb;pirouette; kunnen wetenschappers 
eindelijk op een standaard manier meten in hoeverre een complexer 
soortvormingsmodel de moeite waard is om te gebruiken. 

} % ~\selectlanguage{dutch}
