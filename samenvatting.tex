\chapter*{Samenvatting}
\addcontentsline{toc}{chapter}{Samenvatting}

{\selectlanguage{dutch}

% What is the thesis about?
% What is the purpose of the thesis?
% What were the methods used to research the information?
% What are the results, conclusions, and recommendations that the thesis presents?

\subchapter{Introductie}

\noindent 
\dropcap{D}{eze} Nederlandse samenvatting is speciaal geschreven
voor niet-biologen, zodat ze beter een idee kunnen
krijgen wat er in dit proefschrift besproken wordt.

\subchapter{Soortvorming}

Er zijn op de wereld veel verschillende (dier-, plant-, etc.) soorten.
Helemaal in het begin van het ontstaan van de Aarde, 
was dit nog niet zo, want toen ontstonden de eerste soorten.
In de loop van de tijd zijn er heel veel soorten bijgekomen.
Het proces die dat doet, noemen we soortvorming.

\subchapter{Soortvorming in bacteriën}

Soortvorming kan op meerdere manieren gebeuren.
In bacterieën zeggen we dat twee bacteriën verschillend zijn,
als hun DNA genoeg verschilt. Bacteriën vermenigvuldigen zichzelf
als de omstandigheden gunstig zijn en bij elke celding vinden
er veranderingen in het DNA plaats. Als je twee identieke bacteriën
lang genoeg laat delen, heb je na een tijd twee verschillende 
bacteriesoorten.

\subchapter{Soortvorming in vooral dieren}

Bij dieren is het moeilijker te zeggen wanneer twee dieren
verschillende soorten zijn. Een veelgebruikte definitie is dat
twee groepen dieren verschillende diersoorten zijn, als een kruising
tussen de twee groepen geen of onvruchtbare kleinkinderen oplevert.

Er zijn meerdere mechanismen die ervoor zorgen 
dat soortvorming in dieren plaatsvind. 
Een simpel mechanisme is dat een groep dieren een tweeën gesplits
wordt door een verandering in het landschap, 
zoals een rivier of bergketen.

\subchapter{Fylogenieën}

Als we kijken naar een verzameling soorten over langere tijd (denk
aan miljoenen jaren!), dan zal er waarschijnlijk 
soortvorming plaatsvinden. Sommige soorten zullen meer nieuwe soorten
dan anderen opleveren. We kunnen dit proces laten zien met een
fylogenetische boom.

```
   +----- Mens
-+-+
 | +------Aap
 +--------Dolfijn
```

kun je de evolutionare historie en verwantschap van soorten weergeven.
Deze bomen kunnen niet direct gemeten worden. Inplaats daarvan
worden ze berekend. Achter deze berekening schuilt een wiskundig model
met veel aannames, waaronder de aanname hoe soortvorming werkt.

In die proefschift kijk ik naar het effect van de aanname van hoe soortvorming
werkt. We denken van de meestgebruikte soortvormingsmodellen dat ze simpel
genoeg zijn, maar niet tè simpel. Ik meet in hoeverre dat klopt.

\subchapter{Fylogenieën}

\subchapter{Fylogenieën maken}

In hoofdstuk 2 liet ik \verb;babette; zien: een R package waarmee je BEAST2, 
een Bayesiaans phylogenetisch inferentieprogramma, kunt aanroepen. 
\verb;babette; is een flexibel en robuust programma geworden.

\subchapter{Fylogenetische modellen}

\subchapter{Kijken hoe goed fylogenieën zijn}

In hoofdstuk 3 liet ik \verb;pirouette; zien: een R package waarmee
je kunt meten hoe groot de invloed is van een juist of onjuist 
soortvormingsmodel. Ik en mijn mede-auteur laten zien dat dit instrument naar behoren werkt, mits
je genoeg herhalingen doet om toevalseffecten een kleine rol te laten spelen.

In hoofdstukken 4 en 5 liet ik zien wat de invloed is van het gebruik van een standaard 
soortvormingsmodel als het echte soortvormingsproces stiekum iets complexer is.
In hoofdstuk 4 zijn de echte fylogenetische bomen gesimuleerd met een 
soortvormingsproces waarbij wèl twee soorten tegelijkertijd kunnen ontstaan.
Ik en mijn mede-auteur laten zien dat ...

In hoofdstuk 5 zijn de echte fylogenetische bomen gesimuleerd met een 
soortvormingsproces waarbij soortvorming wèl tijd kost. Ik laat zien
dat ... en dat het het effect van [sampling] ... is.  

\subchapter{Conclusie}

Dit proefschrift leert ons dat ...

Het mooie aan mijn onderzoek is dat andere wetenschappers er zelf gemakkelijk 
ook wat mee kunnen:
zowel \verb;babette; als \verb;pirouette; zijn flexibele en professionele 
R packages. \verb;babette; heeft het mogelijk gemaakt om grootschaliger 
onderzoek te doen aan fylogenetische modellen, doordat nu een experiment
vanuit een script gedaan kan worden, inplaats van handmatig elk 
experiment in te stellen. Met \verb;pirouette; kunnen wetenschappers 
eindelijk op een standaard manier meten in hoeverre een complexer 
soortvormingsmodel de moeite waard is om te gebruiken. 

} % ~\selectlanguage{dutch}
